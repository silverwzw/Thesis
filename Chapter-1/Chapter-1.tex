\chapter{Introduction}
\label{chap:intro}

    When it comes to the file system, what first occurs is usually a disk file system like FAT32 or Ext3. Disk file systems are important because disk was the original persistent storage where we can retrieve and store our valuable data. Their primal goal is to use the disk or other physical media more efficiently. But as a major source of data to the operating system, file systems could have more potential. File systems can be used to access data from remote machine like NFS~\cite{nfs}, can be used to access non-traditional form of data and space like GmailFS~\cite{gmailfs, gmailfs2}, can provide useful and transparent functionalities like encryption~\cite{encrypt}, compression~\cite{compression} or RAID.

    As the network bandwidth increases, we no longer have to keep our data on local disk. Nowadays more and more data are stored online. Individuals are uploading their personal data to web services, commercial companies are using systems like Amazon S3 to store their data remotely. There are a large number of reasons to store data in the cloud. It makes sharing and collaboration easier and ensures that the data can be accessed by its owner anywhere and anytime. Though there are lots of services and applications that help us to access data online. A network file system is able to make this process totally transparent to the user and user programs. The distributed file system can also combine storage on different machines into a logically unified volume while providing features like redundancy and load balancing. Popular distributed file systems adopted by industry and academic institutes include NFS and HDFS. NFS provides the functionality of file sharing and HDFS focuses on MapReduce tasks, etc.

    The NoSQL database is used for store and retrieve data which is modeled differently from the tabular relations. It brings innovation and new ideas to the community, industry and academia. They have become alternatives for classic relational database. Commercial companies like Google and Amazon are using them in production environments for many years. Compared to relational database, they have advantages in performance, scalability and flexibility whereas low functionality, like lack of join query and join update, is their disadvantage. The NoSQL database brings new tools and new ideas to the community and provides an alternative to relational databases. 

    We believe a NoSQL database can serve as a backend to save and manage the actual data. Because a file system usually query for single entity (file, directory) and there is not much demand of join queries and join updates, a NoSQL database that focuses on performance, scalability and flexibility rather than the relational algebra may be a better fit to this task. Also the high scalability of NoSQL database makes it easy to distribute the storage on to different machine.

    Furthermore, when the file system is released from the burden of on disk resource management by using a database as backend, we can put more efforts towards other features of the file system such as snapshot and deduplication. A snapshot file system can be used to test installations, keep track of modifications, provide a rollback mechanism, and make the backup process more efficient. It is becoming more and more popular in modern file system to have the capability to take snapshots. As an example, from FAT to NTFS and from Ext2 to Ext4, more and more file systems have this feature built-in. Ground breaking file system ZFS and experimental file system BTRFS all come with snapshot capability. Deduplication is another popular feature, where the file system try to identify duplicated data and eliminate them in order to save storage space.

\section{Thesis Statement}


\section{Contribution}

    In this thesis, we propose a design of a new distributed file system with snapshot capability. We have the design implemented by Java, MongoDB, FUSE, and other techniques. We also propose several new ways to improve the efficiency of the file system. We tested and compared the performance our implantation with other file systems, studied the snapshot system performance under different scenarios, compare and explain the outcome of the test.
