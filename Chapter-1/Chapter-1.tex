\chapter{Introduction}
\label{chap:intro}

    When it comes to the file system, what first pop into our mind is usually a disk file system like FAT32 or Ext3. Disk file systems was so important because disk was one of the few persistent storage where we can retrieve and store our valuable data. Their algorithm and data structures are designed to use the disk space more efficiently. Most of their effort are invested into disk management like block management and defragmentation. 

    But as the network bandwidth increases, we no longer have to keep our data on local disk. Nowadays more and more data are stored online. Individuals are uploading their personal data to web services, commercial companies are using systems like Amazon S3 to store their data remotely. There are a large number of reasons to store data in the cloud. It makes sharing and collaboration easier and ensures that the data can be accessed by its owner anywhere and anytime. Though there are lots of services and applications that help us to access data online, a file system though network will make this process totally transparent to the user and user programs. The distributed file system can also combine storage on different machine into a logically unified volume and at the same time can provide functionalities like redundancy and load balancing. Popular distributed file system adopted by industry and academic institutes include NFS and HDFS. NFS provides the functionality of file sharing and HDFS focuses on MapReduce tasks, etc.

    We also want to release the file system from the burden of disk resource management. We believe that we can use a database to save and manage the actual data. Compared to the classic relational databases, a NoSQL database which emphasize performance instead of the functionality of relational algebra may be a better fit to this task. Because a file system usually query for single entity (file, directory) so there is not much demand of joint queries and joint updates. Also the high scalability of NoSQL database makes it easy to distribute the storage on to different machine.

    The NoSQL database is used for store and retrieve data which is modeled differently from the tabular relations. It brings innovation and new ideas to the community, industry and academia. They have become alternatives for classic relational database. Commercial companies like Google and Amazon are using them in production environments for many years. Compared to relational database, they have advantages in performance, scalability and flexibility whereas low functionality, like lack of joint query and joint update, is their disadvantage. The NoSQL database brings new tools and new ideas to the community and provides an alternative to relational databases. 

    At last, when the file system is released from the burden of on disk resource management, we can spend more efforts into other features of the file system like increamental backup, transparent encrtption, and snapshot. A snapshot file system can be used to test installations, keep track of modifications, provide a rollback mechanism, and make the backup process more efficient. It is becoming more and more popular in modern file system to have the capability to take snapshots. As an example, NTFS adds snapshot capability whereas its predecessor classic FAT file systems do not support snapshot. The latest Ext4 has a built-in snapshot feature. But ext2 and ext3 does not have native snapshot capability. The ground-breaking file system ZFS on Solaris and the experimental file system BTRFS also come with snapshot feature.

\section{Contribution}

    In this thesis, we propose a design of a new distributed file system with snapshot capability. We implemented the design by using Java, MongoDB, FUSE, and other techniques. We also propose several new ways to improve the effiency of the file system. We tested and compared the performance our implemtation with other file systems, studied the snapshot system performance under different scenrios, compare and explain the outcome of the test.
