\chapter{Related work}
\label{chap:related_work}

    There are important prior works in customized file systems, NoSQL database, and snapshots. Several of them are related to this thesis and they are summarized in this chapter.

\section{POSIX}

    POSIX (Portable Operating System Interface), is a family of IEEE standards for operating systems~\cite{posix_wiki}. It influences the design of many modern operating systems (e.g., Linux, Windows NT and Mac). The POSIX standard defines a standard environment for operating system (process, user, file, directory, etc.) along with a set of APIs for user programs (like fork, exec, I/O functions, etc.)~\cite{posix}. some file format standards (e.g., tar) and some shell utilities (e.g. awk, vi) are also included in the standard. The purpose of this standard is to maintain compatibility across different operating systems such that a user program written for one POSIX operating system will work on any POSIX operating system theoretically. 

    The POSIX standard specifies a set of file system APIs that defines file and directory operations. Any file system that implements this set of APIs should be compatible with operating systems that follows POSIX standard.

\section{FUSE}

    FUSE, the Filesystem in User Space, is a developer framework for file system development. It was originally developed for AVFS (a virtual file system that allows users to look inside an archived or compressed file, or to access remote files)~\cite{avfs} but has become a separate project. It has now been adopted into the Linux kernel and has many ports on other Unix-like operating systems. FUSE provides a programming interface which is very similar to the POSIX file operation standard. A file system with this interface will be able to capture file system calls from kernel module VFS (the Virtual File System)~\cite{vfs}. Such that a user program is able to use standard file system calls to access it as if the file system is supported by the operating system natively. By running the file system in user space, FUSE also isolates the file system from the operating system. In such a way, developers of the file system do not need to understand the kernel code or to debug in the kernel~\cite{fuse}.

    FUSE routes the file system call from VFS in the kernel space back to the user space and then allows the user program to process file system calls. It is a widely used component in Linux file system. A number of well-known projects are using FUSE, for example the OverlayFS, ntfs-3g, and vmware tools. Inspired by FUSE, there are some other user space file system projects like Dokan under windows.

    There are many different language bindings of FUSE. Our implementation is built on the Java bindings `FUSE-JNA'. The `FUSE-JNA' is a recent active project developed by Etienne Perot. The author describes his project as ``No-nonsense, actually-working Java bindings to FUSE using JNA''~\cite{fusejna}. Other Java bindings of FUSE include FUSE-J and jnetfs~\cite{jnetfs}.

\section{MongoDB}

    MongoDB is a document oriented NoSQL distributed database~\cite{mongo_overview}. It focuses on scalability, performance and availability~\cite{mongo_overview}. Document oriented storage that can store semi-structured data makes it flexible and makes it suitable for agile development~\cite{docdb}. In addition, MongoDB also provides features like load balancing and replication. These features make MongoDB an ideal backend for a distributed file system. Developers have already built a file storage system called GridFS using MongoDB which provides a mechanism to store and retrieve a file of any size~\cite{gridfs}.

    MongoDB represents document in JSON (JavaScript Object Notation) format. JSON is a open, human and machine-readable standard that facilitates data interchange. Behind the scenes, MongoDB uses BSON (Binary JSON) to encode and store the documents. Both JSON and BSON format support embedding objects and arrays within other objects and arrays. MongoDB can query and build indexes not only based on top level keys but also based on nested objects. Developers believe this will grant MongoDB users the ease of use and the flexibility together with the speed and richness of a lightweight binary format~\cite{bson}.

    MongoDB is one of the most popular NoSQL databases. It has a large and active developer community. The recent \$150-million investment ensures the long term support and reflects the confidence of investors~\cite{mongInvest}.

\section{Snapshot}

    A snapshot is the state of a system at a particular point in time~\cite{btrfscow}. They can be either read-only or writable. Writable snapshot is also known as clone. This feature is  supported by several advanced file systems like ZFS and BTRFS. The Ext4 file system also has a development branch of writable snapshot. From the user's point of view, a read-only snapshot is an immutable and exact copy of the file system at a specific time, whereas a writable snapshot is a fork of the file system at a particular time spot. A writable snapshot system allows modifications to snapshots but modifications to a snapshot are separated from the origin and other snapshots. (i.e., changes to the snapshot cannot be obsersved by other snapshots or the origin and vice versa) With the help of writable snapshot system, one can easily create and switch between file system branchs and also make changes to them without affecting other branches. Writable snapshots can also provide file system isolation for processes, softwares or virtual machines.

\section{Copy-on-Write}

    Copy-on-write is a strategy widely used in computer science. A program that uses copy-on-write strategy accesses data though a pointer or a reference. When a copy operation is requested, instead of making a copy of actual data, a program that uses copy-on-write strategy will simply return a new reference to the original data. Only when the first modification to one of the ``copies'' is requested, the program will then make an actual copy of the original data and then apply the modification to that copy. At the end, the program will point the corresponding pointer or reference to the updated copy. This design not only eliminates unnecessary overhead in data copy but also ensures consistency, integrity and an easy support of transactions. In addition to those benefits, a file system using copy-on-write strategy will be able to take snapshots with low overhead. Without copy-on-write strategy, snapshots either write in-place which is more expensive, or require special architectures in storage system like the Split-Mirror architecture.

\section{Existing Snapshot Storage Systems}

    Snapshot is an important feature of a storage system. Existing works include LVM snapshot, SnapFS, OverlayFS, ZFS, BTRFS, and Ext4.
    
    The LVM snapshot is a snapshot system included in the logical volume manager~\cite{lvm, disk_perform_lvm}. It takes snapshots of blocks in the logical volume and gives snapshots capability to any file system built upon the LVM system. However, as an underlying service, the LVM snapshot system will not be aware of file structures, making it hard to find duplicate data stream and thus is less efficient in terms of disk spaces.
    
    SnapFS is a file system focusing on snapshots in the Linux kernel~\cite{snapfs}. It is strongly coupled with the underlying file system. Therefore, SnapFS is able to apply copy-on-write strategy on block but restricts its underlying file system to the Ext file system family. 
    
    OverlayFS~\cite{overlayfs}, also called UnionFS, is a file system popular in embedded systems, hand held devices, PDAs and smartphones. Similar to SnapFS, the OverlayFS is also built upon other file systems. But OverlayFS does not restrict the type of its underlying file system as SnapFS does. This gives OverlayFS a lot flexibility as it can adapt to any storage media that has a standard file system implemented. The OverlayFS applies copy-on-write on file and directory level.

    BTRFS~\cite{btrfs} is an experimental file system developed by Oracle. It was inspired by the well-known Solaris file system ZFS and shares a lot of similarities with ZFS. Both of them use copy-on-write snapshots at the block level granularity~\cite{btrfscow}.
    
    The Ext4 file system was extended to include a snapshot subsystem that applies copy-on-write strategy on block level. The writable snapshot feature for Ext4 is currently under development by the same group of developers~\cite{ext4snap}. It is inspired by the CTERA Network's NEXT3 file system which is a clone of ext3 file system with built-in support for snapshots.

\section{The Rsync Algorithm}
    
    The rsync algorithm is used in the rsync utility in Unix-like system to synchronize files though the network. This algorithm was invented by Andrew Tridgell and he described the algorithm as ``for the efficient update of data over a high latency and low bandwidth link''~\cite{rsync_alg}. The algorithm calculates and compares a set of weak checksums of blocks between a remote file and a local file to identify duplicated blocks. In order to achieve efficiency, the checksum algorithm takes only O(1) time complexity to calculate based on prior results, such that locating duplication efficiently at any offset in a local file becomes possible. The algorithm performs well with delta encoding and reduces the data transferred between remote machines.

